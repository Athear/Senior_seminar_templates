% This is a sample document using the University of Minnesota, Morris, Computer Science
% Senior Seminar modification of the ACM sig-alternate style. Much of this content is taken
% directly from the ACM sample document illustrating the use of the sig-alternate class. Certain
% parts that we never use have been removed to simplify the example, and a few additional
% components have been added.

% See https://github.com/UMM-CSci/Senior_seminar_templates for more info and to make
% suggestions and corrections.

\documentclass{sig-alternate}
\usepackage{color}

%%%% User-defined macros
\newcommand{\lam}{\lambda}
\newcommand{\mycomment}[1]{\textcolor{red}{#1}}
%%%%% Uncomment the following line and comment out the previous one
%%%%% to remove all comments
%%%%% NOTE: comments still occupy a line even if invisible;
%%%%% Don't write them as a separate paragraph
%\newcommand{\mycomment}[1]{}

\begin{document}

% --- Author Metadata here ---
%%% REMEMBER TO CHANGE THE SEMESTER AND YEAR
\conferenceinfo{UMM CSci Senior Seminar Conference, December 2013}{Morris, MN}

\title{System of Programming Language Interoperability}

\numberofauthors{1}
\author{
% The command \alignauthor (no curly braces needed) should
% precede each author name, affiliation/snail-mail address and
% e-mail address. Additionally, tag each line of
% affiliation/address with \affaddr, and tag the
% e-mail address with \email.
\alignauthor
Todd Malone\\
	\affaddr{Division of Science and Mathematics}\\
	\affaddr{University of Minnesota, Morris}\\
	\affaddr{Morris, Minnesota, USA 56267}\\
	\email{malon153@morris.umn.edu}
}

\maketitle
\begin{abstract}
A discussion of systems and methods for the interop of programming languages
\end{abstract}

% A category with the (minimum) three required fields
\category{H.4}{Information Systems Applications}{Miscellaneous}
%A category including the fourth, optional field follows...
\category{D.2.8}{Software Engineering}{Metrics}[complexity measures, performance measures]

\terms{Delphi theory}

\keywords{ACM proceedings, \LaTeX, text tagging}

\section{Introduction}
Introduction will introduce what Interop is, what it means to be interoperable, and
why it is important to people.

\section{Interoperability}\label{difficulties}
\mycomment{this section needs a better name}


\subsection*{Requirements for Interop}
In order for interop to be a viable option, the languages in question must have some common ground. At at basic level
this means they should be able to act on similar data types and have access to the same types of files. However, simply dealing with data of the same type does not guarantee successful data sharing, as languages may deal with similar data differently. Ide and Pustejovsky \cite{Ide:2010} suggest a method for dealing with potential type conflicts through the use of metadata. Metadata can describe the type and classification of data in more broadly applicable and translatable fashion.

Of importance in every level of the transaction is standardization. Metadata, as mentioned, can be a useful tool for data interoperability. In order to actually use it, however, both languages or programs must agree on what tags will represent what data. Standardized metadata sets ensure that programs not built with each other in mind can still interoperate without being fully redesigned. There are several catalogs attempting to form full metadata tag sets, such as the Open Archives Initiative (OAI), the ELRA Universal Catalog, and the NICT Shachi catalog.\cite{Ide:2010}

\subsection*{Difficulties in achieving Interop}
Despite tools available to promote interoperability, there are still difficulties inherent to building an interoperative system. Metadata tags can be a powerful too, but there are some subtleties about languages that should be taken into account.

One of these is the type systems of the languages in question. For instance, converting between the weakly-typed language Groovy and the strongly-type Java might require that all of the Groovy data be labeled, or it's arrays be of only a single type, effectively forcing Groovy into a strong type system. \mycomment{This requires some better introduction. also, look up how groovy does manage its Java interop}

Another area of concern is adherence to standardization.
As evidenced above, there exist a wide array of possible standards to use. Conformance to one standard does not guarantee connectivity with a system using a different standard, reducing the usefulness of such standards by the number of them in existence.\cite{?}



\section{Virtual Machine Systems}\label{VM}
Here, I will detail how VMs handle the issue of multiple programming languages.
From my current understanding, this is mainly done through a shared low-level language, such as Java's bytecode, or the .NET Common Intermediate Language.


\subsection*{Single Platform Language Interoperability}
Virtual machines can be used as a language interoperability platform when the system in question is meant to be run on a single machine. 

\subsection*{Platform Portability}


\section{Markup Language Solutions}\label{ML}



\subsection*{Distributed Connectivity}
Markup languages are primarily useful when dealing with systems that span multiple servers or platforms. \mycomment{Need more here}

Markup languages, by their heavily tagged structure, \footnote{\mycomment{EXPLAIN THE HECK OUT OF THIS}} are a perfect match for Ide and Pustejovsky's model of metadata interoperability.\cite{Ide:2010} The tagging system used to describe aspects of a markup language can be adapted to describe data types within other languages and data models.

One system that makes use of this is the Starlink framework\cite{Bromberg:2011}. Starlink's purpose is to facilitate passing messages between languages that do not share a common communication protocol. As such, for each language it must model both the incoming message and the protocols for each language.
The translation of the message between representations is handled using sets of logical automata, while the representations themselves are done using a specialized markup language they call the Message Description Language. In an MDL, each piece of data is contained within its own set of ML tags , referred to as primitive fields. Primitive fields are themselves contained in another set of tags, called structured fields. These fields can be likened to the metadata labels, with primitive fields standing for syntactic level labels and structured fields standing for semantic level labels. \mycomment{I'm not sure that's strictly true, so look into this}.
\mycomment{Consider including figure 7 of \cite{Bromberg:2011} here}

\mycomment{Consider using: MDLs are set up in a specific way so that the automata are able to efficiently translate between two MDL representations}

%%TODO: discuss Starlink and its use of MDL to enhance multi-system interoperability

\subsection*{FML and Hardware Independence}
%%here, discuss FML and its purpose of freeing the design of fuzzy controllers from system and hardware constraints

Markup languages can also be used to divorce programs from hardware dependencies, freeing them to be more interoperable over distributed platforms.

\mycomment{Will need some description of what's nice about fuzzy controllers}
The Fuzzy Markup Language (FML) is a prime example of this. At present, the best method for implementing fuzzy controllers for a particular system has involved designing specific hardware models based on the system it was being designed for. When dealing with a number of platforms using different hardware, this meant different designs were required to implement the same controller for each. FML attempts to shift the control of fuzzy logic from hardware to software, such that it can be implemented on multiple platforms for little additional cost.

\mycomment{Consider using \cite{Acampora:2013} figure 1 to explain the fuzzy system}
Once again, 
While significantly more complicated, FML can be described in a basic sense to be using the XML tags as a means of attaching metadata to data involved in a fuzzy control system. It is more complicated because they use more than the two levels of metadata described by Ide and Pustejovsky. FML is meant to support a fairly complex system, and the models used within that system for determining \textcolor{red}{stuff}. As such, they go beyond modeling primitive data to modeling full data structures. It goes beyond this, but out of the realm of metadata. 

\section{Conclusions}


\section{Acknowledgments}


% The following two commands are all you need in the
% initial runs of your .tex file to
% produce the bibliography for the citations in your paper.
\bibliographystyle{abbrv}
% sample_paper.bib is the name of the BibTex file containing the
% bibliography entries. Note that you *don't* include the .bib ending here.
\bibliography{bibliography}
% You must have a proper ".bib" file
%  and remember to run:
% latex bibtex latex latex
% to resolve all references

\end{document}
