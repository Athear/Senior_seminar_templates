% This is a sample document using the University of Minnesota, Morris, Computer Science
% Senior Seminar modification of the ACM sig-alternate style. Much of this content is taken
% directly from the ACM sample document illustrating the use of the sig-alternate class. Certain
% parts that we never use have been removed to simplify the example, and a few additional
% components have been added.

% See https://github.com/UMM-CSci/Senior_seminar_templates for more info and to make
% suggestions and corrections.

\documentclass{sig-alternate}
\usepackage{color}

%%%% User-defined macros
\newcommand{\lam}{\lambda}
\newcommand{\mycomment}[1]{\textcolor{red}{#1}}
%%%%% Uncomment the following line and comment out the previous one
%%%%% to remove all comments
%%%%% NOTE: comments still occupy a line even if invisible;
%%%%% Don't write them as a separate paragraph
%\newcommand{\mycomment}[1]{}

\begin{document}

% --- Author Metadata here ---
%%% REMEMBER TO CHANGE THE SEMESTER AND YEAR
\conferenceinfo{UMM CSci Senior Seminar Conference, December 2013}{Morris, MN}

\title{System of Programming Language Interoperability}

\numberofauthors{1}

\author{
% The command \alignauthor (no curly braces needed) should
% precede each author name, affiliation/snail-mail address and
% e-mail address. Additionally, tag each line of
% affiliation/address with \affaddr, and tag the
% e-mail address with \email.
\alignauthor
Todd Malone\\
	\affaddr{Division of Science and Mathematics}\\
	\affaddr{University of Minnesota, Morris}\\
	\affaddr{Morris, Minnesota, USA 56267}\\
	\email{malon153@morris.umn.edu}
}

\maketitle
\begin{abstract}
\mycomment{A discussion of systems and methods for the interop of programming languages}
\end{abstract}

% A category with the (minimum) three required fields
%\category{H.4}{Information Systems Applications}{Miscellaneous}
%A category including the fourth, optional field follows...
%\category{D.2.8}{Software Engineering}{Metrics}[complexity measures, performance measures]

%\terms{Delphi theory}

%\keywords{ACM proceedings, \LaTeX, text tagging}

\section{Introduction}

This paper will discuss interoperability of programming languages. I'll be discussing why interoperability is desirable, followed by what challenges are presented in achieving interop.
I'll discuss two particular approaches to interoperability in markup languages(MLs) and virtual machines (VMs), with focus on one or two particular examples (CLR/.NET for VMs and Starlink or FML for MLs).


The introduction will primarily introduce what interoperability is and what it means for languages to be interoperable \cite{Ide:2010} \mycomment{(different between VMs and MLs?)}

\section{Interoperability}\label{difficulties}
This section will discuss both why interoperability is important and hard.
Importance will cover things like:
\begin{itemize}
	\item problem domain: languages have libraries and capabilities build for specific problems. This gives them strength in that area (such as SQL database queries?), but often they have weaknesses in other areas (such as Erlang's strings).
	\item level(?): some languages have finer-grained control over storage (like the C family), while others deal with more abstract and generalize-able data structures (like... java or ruby).
\end{itemize}

Difficulties will cover:
\begin{itemize}
	\item data type conversion and interpretation \cite{Chisnall:2013}
	\item loss of flexibility or information due to lowest-common-denominator type restriction 
	\item standardization: Is there a standard? If there is, how closely is it followed? Are there competing standards? \cite{Shetty:2009, Kats:2010}
	\item error handling
\end{itemize}

It may be a good idea to talk about tags and metadata in a subsection here, or a new section entirely. Both MLs and VMs make extensive use of metadata as a method of data translation.

\section{Virtual Machine Systems}\label{VM}
This section will cover how virtual machines handle some of the issues discussed above.

\section{Markup Language Solutions}\label{ML}
This section will be about markup languages, and how they contribute to solutions to interoperability

%\section{Conclusions}


%\section{Acknowledgments}


% The following two commands are all you need in the
% initial runs of your .tex file to
% produce the bibliography for the citations in your paper.
\bibliographystyle{abbrv}
% sample_paper.bib is the name of the BibTex file containing the
% bibliography entries. Note that you *don't* include the .bib ending here.
\bibliography{bibliography}  
% You must have a proper ".bib" file
%  and remember to run:
% latex bibtex latex latex
% to resolve all references

\end{document}
