% This is a sample document using the University of Minnesota, Morris, Computer Science
% Senior Seminar modification of the ACM sig-alternate style. Much of this content is taken
% directly from the ACM sample document illustrating the use of the sig-alternate class. Certain
% parts that we never use have been removed to simplify the example, and a few additional
% components have been added.

% See https://github.com/UMM-CSci/Senior_seminar_templates for more info and to make
% suggestions and corrections.

\documentclass{sig-alternate}
\usepackage{color}

%%%% User-defined macros
\newcommand{\lam}{\lambda}
\newcommand{\mycomment}[1]{\textcolor{red}{#1}}
%%%%% Uncomment the following line and comment out the previous one
%%%%% to remove all comments
%%%%% NOTE: comments still occupy a line even if invisible;
%%%%% Don't write them as a separate paragraph
%\newcommand{\mycomment}[1]{}

\begin{document}

% --- Author Metadata here ---
%%% REMEMBER TO CHANGE THE SEMESTER AND YEAR
\conferenceinfo{UMM CSci Senior Seminar Conference, December 2013}{Morris, MN}

\title{System of Programming Language Interoperability}

\numberofauthors{1}

\author{
% The command \alignauthor (no curly braces needed) should
% precede each author name, affiliation/snail-mail address and
% e-mail address. Additionally, tag each line of
% affiliation/address with \affaddr, and tag the
% e-mail address with \email.
\alignauthor
Todd Malone\\
	\affaddr{Division of Science and Mathematics}\\
	\affaddr{University of Minnesota, Morris}\\
	\affaddr{Morris, Minnesota, USA 56267}\\
	\email{malon153@morris.umn.edu}
}

\maketitle
\begin{abstract}
\mycomment{A discussion of systems and methods for the interop of programming languages}
\end{abstract}

% A category with the (minimum) three required fields
\category{H.4}{Information Systems Applications}{Miscellaneous}
%A category including the fourth, optional field follows...
\category{D.2.8}{Software Engineering}{Metrics}[complexity measures, performance measures]

\terms{Delphi theory}

\keywords{ACM proceedings, \LaTeX, text tagging}

\section{Introduction}
Introduction will introduce what Interop is, what it means to be interoperable, and
why it is important to people.

\section{Achieving Interoperability}\label{difficulties}
\mycomment{this section needs a better name}
Section will discuss why interoperability is hard.
To a large degree, this will concern 
\begin{itemize}
	\item data-level (Syntactic and Semantic)
	\item communication level (protocols)
	\item some other things, involving systems or networks.
\end{itemize}

\section{Virtual Machine Systems}\label{VM}
Here, I will detail how VMs handle the issue of multiple programming languages.
From my current understanding, this is mainly done through a shared low-level language \footnote{such as Java's bytecode, or the .NET Common Intermediate Language}

\section{Markup Language Solutions}\label{ML}
This section will be about Markup Languages, and what they contribute to the discussion of interoperability.
Presently, I only know of one full system\footnote{Starlink} that uses a markup language at its core. I'll try to discuss Starlink's use of ML here,
but other MLs should be discussed, even if I don't have distinct systems to cite.

\section{Conclusions}
This paragraph will end the body of this sample document.
Remember that you might still have Acknowledgments or
Appendices; brief samples of these
follow.  There is still the Bibliography to deal with; and
we will make a disclaimer about that here: with the exception
of the reference to the \LaTeX\ book, the citations in
this paper are to articles which have nothing to
do with the present subject and are used as
examples only.

\section{Acknowledgments}

This section is optional; it is a location for you
to acknowledge grants, funding, editing assistance and
what have you.

It is common (but by no means necessary) for students to thank
their advisor, and possibly other faculty, friends, and family who provided
useful feedback on the paper as it was being written.

In the present case, for example, the
authors would like to thank Gerald Murray of ACM for
his help in codifying this \textit{Author's Guide}
and the \textbf{.cls} and \textbf{.tex} files that it describes.

% The following two commands are all you need in the
% initial runs of your .tex file to
% produce the bibliography for the citations in your paper.
\bibliographystyle{abbrv}
% sample_paper.bib is the name of the BibTex file containing the
% bibliography entries. Note that you *don't* include the .bib ending here.
\bibliography{sample_paper}  
% You must have a proper ".bib" file
%  and remember to run:
% latex bibtex latex latex
% to resolve all references

\end{document}
