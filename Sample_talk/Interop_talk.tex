\documentclass{beamer}

\mode<presentation>
{
  \usetheme{Antibes}
  \setbeamercovered{transparent}
}

\usepackage[english]{babel}
\usepackage[latin1]{inputenc}
\usepackage{times}
\usepackage[T1]{fontenc} 
% Or whatever. Note that the encoding and the font should match. If T1
% does not look nice, try deleting the line with the fontenc.
\usepackage{amsmath}

\newcommand{\linespace}{\vskip 0.25cm}

\definecolor{MyForestGreen}{rgb}{0,0.7,0} 
\newcommand{\tableemph}[1]{{#1}}
\newcommand{\tablewin}[1]{\tableemph{#1}}
\newcommand{\tablemid}[1]{\tableemph{#1}}
\newcommand{\tablelose}[1]{\tableemph{#1}}

\definecolor{MyLightGray}{rgb}{0.6,0.6,0.6}
\newcommand{\tabletie}[1]{\color{MyLightGray} {#1}}

% The text in square brackets is the short version of your title and will be used in the
% header/footer depending on your theme.
\title[Interoperability]{Interoperability in Programming Languages}

% Sub-titles are optional - uncomment and edit the next line if you want one.
% \subtitle{Why does sub-tree crossover work?} 

% The text in square brackets is the short version of your name(s) and will be used in the
% header/footer depending on your theme.
\author[Malone]{Todd Owen Malone}

% The text in square brackets is the short version of your institution and will be used in the
% header/footer depending on your theme.
\institute[U of Minn, Morris]
{
  Division of Science and Mathematics \\
  University of Minnesota, Morris \\
  Morris, Minnesota, USA
}

% The text in square brackets is the short version of the date if you need that.
\date[April '14, Senior Sem] % (optional)
{28 April 2014 \\ Senior Seminar}

% Delete this, if you do not want the table of contents to pop up at
% the beginning of each subsection:
\AtBeginSection[]
{
  \begin{frame}<beamer>
    \frametitle{Outline}
    \tableofcontents[currentsection, hideothersubsections]
  \end{frame}
}

\begin{document}

\begin{frame}
  \titlepage
\end{frame}

% For a 20-25 minute senior seminar talk you probably want something like:
% - Two or three major sections (other than the summary).
% - At *most* three subsections per section.
% - Talk about 30s to 2min per frame. So there should probably be between
%   15 and 30 frames, all told.

\section*{Interoperability}

\subsection*{Introduction to Interop}

\begin{frame}
  \frametitle{What is Interop?}
  
  \begin{columns}
  \begin{column}{0.6\textwidth}
  \begin{itemize}
  	\item Interoperability: The ability for a system to use parts from another system
	\item In programming languages: A program written in one language can use a program in a different language
	\item 
  \end{itemize}
  \end{column}
  \begin{column}{0.4\textwidth}
   \includegraphics[width=0.95\textwidth]{Illustrations/Empty_cocoon_crop_by_Bluedrakon_from_Flickr.jpg}
       \\
    \only{\tiny{Bluedrakon \\ \url{http://tr.im/pWUi} }}
  \end{column}
  \end{columns}
\end{frame}

\begin{frame}
  \frametitle{Why is Interop Important?}
  
  \begin{columns}
  \begin{column}{0.6\textwidth}
  \begin{itemize}
  	\item Third-party systems: source code is unavailable
	\item Legacy systems: 
  \end{itemize}
  \end{column}
  
  \begin{column}{0.6\textwidth}
  \begin{itemize}
  	\item Language Purpose
  	\item C has fine-grain memory access
  	\item 
  \end{itemize}
  \end{column}
  \end{columns}
\end{frame}

\subsection*{Outline}

\begin{frame}
  \frametitle{Outline}
  \tableofcontents[hideallsubsections]
\end{frame}

\section[Interop Tools]{Tools used in achieving Interoperability}

\subsection{Virtual Machines}

\begin{frame}
  \frametitle{Virtual Machines}
  
  \begin{columns}
  \begin{column}{0.6\textwidth}
  \begin{itemize}
  	\item 
	\item 
	\item 
	\item 
	\item 
  \end{itemize}
  \end{column}
  \end{columns}
\end{frame}

\subsection{Markup Languages}

\begin{frame}
  \frametitle{Markup Languages}
  
  \begin{columns}
  \begin{column}{0.6\textwidth}
  \begin{itemize}
  	\item 
	\item 
	\item 
	\item 
	\item 
  \end{itemize}
  \end{column}
  \end{columns}
\end{frame}

\section[Difficulties and Approaches]{Difficulties in Interop and approaches to dealing with them}

\subsection{Overview}

\begin{frame}
  \frametitle{Some common difficulties in interop}
  
  \begin{columns}
  \begin{column}{0.6\textwidth}
  \begin{itemize}
  	\item 
	\item 
	\item 
	\item 
	\item 
  \end{itemize}
  \end{column}
  \end{columns}
\end{frame}

\subsection{Metadata}

\begin{frame}
  \frametitle{Metadata and type conversion}

	Metadata: Data about data
	
	{\tt (def mylist [1, 2, 3, 4])}
	
	{\tt (with-meta mylist \{:length 4, :type Integer\})}
\end{frame}

\subsection{Standards}

\begin{frame}
  \frametitle{The importance of Standards}
  
  \begin{columns}
  \begin{column}{0.6\textwidth}
  \begin{itemize}
  	\item 
	\item 
	\item 
	\item 
	\item 
  \end{itemize}
  \end{column}
  \end{columns}
\end{frame}

\section[Conclusions]{Conclusions}

\begin{frame}
  \frametitle{Conclusions}
  
  \begin{columns}
  \begin{column}{0.6\textwidth}
  \begin{itemize}
  	\item 
	\item 
	\item 
	\item 
	\item 
  \end{itemize}
  \end{column}
  \end{columns}
\end{frame}




\begin{frame}
	\frametitle{The End!}
	
	
		
	\linespace
	\linespace
	
	Contact:  
	\begin{itemize}
		\item \texttt{malone153@morris.umn.edu}
	\end{itemize}
	
	\linespace
	\linespace
	
	\begin{center}
	{\huge Questions?}
	\end{center}
\end{frame}

\section*{References}

\begin{frame} 
	\frametitle{References} 
	
	\begin{thebibliography}{lskdjf}
	
	\bibitem{McPhee:2009:gecco}
N.~F. McPhee, E.~Crane, S.~Lahr, and R.~Poli.
\newblock Developmental Plasticity in Linear Genetic Programming.
\newblock In G\"unther Raidl, \emph{et al}, editors, {\em GECCO '09}, pages 1019--1026, Montr\'eal, Qu\'ebec, Canada, 2009.
	
	\bibitem{citeulike:3452411}
	R.~Poli and N.~McPhee.
\newblock A linear estimation-of-distribution {GP} system.
\newblock In M.~O'Neill, \emph{et al}, editors, {\em EuroGP 2008}, volume
  4971 of {\em LNCS}, pages 206--217, Naples,
  26-28 Mar. 2008. Springer.
  
  	\end{thebibliography}
	
	\linespace
	\begin{center}
	See the GECCO '09 paper for additional references.
	\end{center}
\end{frame} 

\end{document}


