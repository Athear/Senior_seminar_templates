% This is a sample document using the University of Minnesota, Morris, Computer Science
% Senior Seminar modification of the ACM sig-alternate style to generate a simple annotated
% bibliography. The idea is that this document is fairly short, consisting of a brief description
% of your sources and how you intend to use them (or not). Most of the ``content'' of the
% generated document comes from the bibliography file, including the notes field which will
% provide the annotations.

% See https://github.com/UMM-CSci/Senior_seminar_templates for more info and to make
% suggestions and corrections.

\documentclass{sig-alternate}

\begin{document}

% --- Author Metadata here ---
%%% REMEMBER TO CHANGE THE SEMESTER AND YEAR
\conferenceinfo{UMM CSci Senior Seminar Conference, December 2013}{Morris, MN}

\title{Interoperability Systems}

\numberofauthors{1}

\author{
% The command \alignauthor (no curly braces needed) should
% precede each author name, affiliation/snail-mail address and
% e-mail address. Additionally, tag each line of
% affiliation/address with \affaddr, and tag the
% e-mail address with \email.
\alignauthor
Todd Malone\\
	\affaddr{Division of Science and Mathematics}\\
	\affaddr{University of Minnesota, Morris}\\
	\affaddr{Morris, Minnesota, USA 56267}\\
	\email{malon153@morris.umn.edu}
}

\maketitle

\keywords{ACM proceedings, \LaTeX, text tagging}

\section{On Interop: Focus and Goals}
This paper will discuss interoperability of programming languages and what is involved in making that happen.
Some steps along the way may include:

\begin{itemize}
\item A general exploration, utilizing \cite{ide:2010, Bromberk:2011, Matthews:2009}.
\item An exploration of Virtual Machines, with \cite{Shetty:2009, Chen:2010, Li:2013}. This section may also include wiki references.
\item A look at Markup Languages as a method of realizing interop, with examples from \cite{Acampora:2013,Bromberk:2011}. I will need more papers here. 
\end{itemize}

I have a few papers with uncertain utility\cite{Kats:2010,Matthews:2009}, but they may still be useful.
Apart from these, I will need a few more papers for reference in most areas.

% The following two commands are all you need to
% produce the bibliography for the citations in your paper.
\bibliographystyle{abbrv}
% annotated_bibliography.bib is the name of the BibTex file containing 
% all the bibliography entries for this example. Note that you *don't* include the .bib ending
% in the \bibliography command.
\bibliography{bibliography}  

% You must have a ".bib" file and remember to run:
%     pdflatex bibtex pdflatex pdflatex
% in order to see all the citation references correctly.

\end{document}



